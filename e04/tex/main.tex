\documentclass{article}
\usepackage[utf8]{inputenc}
\usepackage[english]{babel}
\usepackage{fullpage} % Package to use full page
\usepackage{parskip} % Package to tweak paragraph skipping
\usepackage{tikz} % Package for drawing
\usepackage{amsmath}
\usepackage{hyperref}
\usepackage{enumitem}
\usepackage{graphicx}
\usepackage{pdfpages}
\usepackage{pgfplots} 
\usepackage{pgf}
\pgfplotsset{compat=1.14}
\usetikzlibrary{arrows, automata, positioning}
\usetikzlibrary{decorations.pathmorphing}
\usepackage{subcaption}
\usepackage{diagbox}

%%\usepackage[demo]{graphicx}
%\usepackage[labelformat=parens]{subfig}
%\usepackage{caption}
%%\usepackage[inline]{enumitem}

\usepackage{booktabs}
\usepackage{longtable}
\usepackage{listings}
\lstset{language=Python}


\newcommand{\exnum}{04} % Enter the problem set number here!
\newcommand{\tilda}{{\raise.17ex\hbox{$\scriptstyle\mathtt{\sim}$}}}

\newcounter{problem}[section]
\newenvironment{prob}[1]
{
    \refstepcounter{problem}
    \Large{\textbf{Problem \exnum.\theproblem}  \qquad \textit{#1}}
    \begin{enumerate}[label=\alph*]
    \normalsize
}{
    \end{enumerate}
}


\title{Exercise \exnum \\ 
    Molecular Dynamics 2019}
\author{Benjamin Kurt Miller and Lauren Green}
\date{\today}

\begin{document}
\maketitle


\begin{prob}{Download a PDB file and visualize it in VMD}
\item I downloaded the Trimeric Structure of Langerin known in RCSB PDB as 3p5g. The pdb was cleaned for presentation using the following commands. This should leave a single "chain" or "protein monomer" without water, ions, or extraneous information as requested in the assignment.

\lstset{language=bash}
\begin{lstlisting}
grep "B" 3p5g.pdb > chain_b.pdb
grep -v "HOH" chain_b.pdb > chain_b.pdb
grep -v "HETATM" chain_b.pdb > chain_b.pdb
\end{lstlisting}

The requested procedure was followed and the rendered image from VMD is included in Figure \ref{fig:render}.

\begin{figure}[!ht]
	\centering
	\includegraphics[width=.5\linewidth]{../3p5g/3p5g.png}
	\caption{Chain B rendered in VMD.}
	\label{fig:render}
\end{figure}

\end{prob}

\begin{prob}{Structure minimization}
	\item 
\end{prob}

\begin{prob}{NVE ensemble}
	\item 
\end{prob}

\begin{prob}{NVT ensemble}
	\item 
\end{prob}

\begin{prob}{NPT ensemble}
	\item 
\end{prob}

\begin{prob}{Production MD run}
	\item 
\end{prob}

\end{document}
