\documentclass{article}
\usepackage[utf8]{inputenc}
\usepackage[english]{babel}
\usepackage{fullpage} % Package to use full page
\usepackage{parskip} % Package to tweak paragraph skipping
\usepackage{tikz} % Package for drawing
\usepackage{amsmath}
\usepackage{hyperref}
\usepackage{enumitem}
\usepackage{graphicx}
\usepackage{pdfpages}
\usepackage{pgfplots} 
\usepackage{pgf}
\pgfplotsset{compat=1.14}
\usetikzlibrary{arrows, automata, positioning}
\usetikzlibrary{decorations.pathmorphing}
\usepackage{subcaption}
\usepackage{diagbox}

\newcommand{\exnum}{02} % Enter the problem set number here!
\newcommand{\tilda}{{\raise.17ex\hbox{$\scriptstyle\mathtt{\sim}$}}}

\newcounter{problem}[section]
\newenvironment{prob}[1]
{
    \refstepcounter{problem}
    \Large{\textbf{Problem \exnum.\theproblem}  \qquad \textit{#1}}
    \begin{enumerate}[label=\alph*]
    \normalsize
}{
    \end{enumerate}
}


\title{Exercise \exnum \\ 
    Molecular Dynamics 2019}
\author{Benjamin Kurt Miller}
\date{\today}

\begin{document}
\maketitle


\begin{prob}{Energy minimisation of force fields}
\item Gradient and Hessian are defined as 

\begin{align*}
    \nabla f_{i} &= \frac{\partial f}{\partial x_{i}} \\
    \mathbf{H}_{i,j}(f) &= \frac{\partial^{2} f}{\partial x_{i} \partial x_{j}}
\end{align*}

respectively. Given our function $U(x, y)$, the function, gradient, and hessian matrix are defined

\begin{align}
    U(x,y) &= (x - y)^{4} + 2x^{2} + y^{2} - x + 2y \\
    \nabla U(x,y) &=
    \begin{pmatrix}
        4(x-y)^{3} + 4x - 1 & -4(x-y)^{3} + 2y + 2 \\
    \end{pmatrix}^{T} \\
    \mathbf{H}(U(x,y)) &= 
    \begin{pmatrix}
        12(x-y)^{2} + 4 & -12(x-y)^{2} \\
        -12(x-y)^{2} & 12(x-y)^{2} + 2 \\
    \end{pmatrix}.
\end{align}

\item 

\begin{figure}
	\begin{subfigure}{.5\textwidth}
		\centering
		\includegraphics[width=.8\linewidth]{../grad_descent_00.png}
		\caption{1a}
		\label{fig:sfig1}
	\end{subfigure}%
	\begin{subfigure}{.5\textwidth}
		\centering
		\includegraphics[width=.8\linewidth]{../grad_descent_11.png}
		\caption{1b}
		\label{fig:sfig2}
	\end{subfigure}
	\caption{plots of....}
	\label{fig:fig}
\end{figure}

\begin{center}
	\begin{tabular}{||c c c c||} 
		\hline
		Col1 & Col2 & Col2 & Col3 \\ [0.5ex] 
		\hline\hline
		1 & 6 & 87837 & 787 \\ 
		\hline
		2 & 7 & 78 & 5415 \\
		\hline
		3 & 545 & 778 & 7507 \\
		\hline
		4 & 545 & 18744 & 7560 \\
		\hline
		5 & 88 & 788 & 6344 \\ [1ex] 
		\hline
	\end{tabular}
\end{center}






\end{prob}

\end{document}
